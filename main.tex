\documentclass{article}
\usepackage{bubblecv}

\begin{document}

%\pagecolor{lightgray}


\begin{cv}[avatar]{Gustav Svennas}{Nyexaminerad datastudent}


\cvsection[summary]{Profil}  %-----------------------------------------------------------
Som person är jag i första hand sport- och teknikintresserad och jag gillar att ha en varierande och aktiv fritid. Jag har tidigare blivit beskriven som tålmodig, uppmärksam och hjälpsam. Jag är social och arbetar bra både ensam och i grupp, något jag fått från att ha spelat hockey i snart 20 år. Jag är arbetsvillig, kan ta kritik och försöker ofta se det positiva i situationer, vilket brukar göra mig enkel att arbeta med.


\cvsection[work]{Arbetslivserfarenhet} %------------------------------------------------------

\begin{cvevent}[juli 2022][aug. 2022]
    \cvname{Kontorsservice}
    \cvdescription{Riksidrottsförbundet, Stockholm}
    \begin{itemize}
        \item Ansvarig för underhåll av kontoret.
    \end{itemize}
\end{cvevent}


\begin{cvevent}[mars 2021][okt. 2021]
    \cvname{Uppdatering av inventariesystemen}
    \cvdescription{Klarna (IT-konsult via Academic Work), Stockholm}
    \begin{itemize}
        \item Ansvar för att uppdatera flera inventariesystem för anställdas datorer, mobiler och andra enheter.
        \item Samarbetade och hade kontakt med kollegor från bland annat USA, Tyskland och Australien.
    \end{itemize}
\end{cvevent}


\begin{cvevent}[juni 2015][juli 2015]
    \cvname{Produktionsarbetare}
    \cvdescription{Maquet Critical Care AB, Solna}
    \begin{itemize}
        \item Hjälpte till att förbereda delar för ihopsättning av diverse medicinteknisk utrustning.
    \end{itemize}
\end{cvevent}
%------------------------------------------------------

\cvsection[education]{Utbildning}  %------------------------------------------------------

\begin{cvevent}[2018][2023]
    \cvname{Kandidatprogrammet, Datateknik}
    \cvdescription{Linköpings Universitet (studiepaus 2021)}
    \textbf{Kurser:}
    \begin{itemize}
        \item Digitalteknik, datorkonstruktion, programutvecklingsmetodik, AI, cybersäkerhet m.m.
        \item Objektorienterad- och processprogrammering. 
        \item Datastrukturer, algoritmer, programmeringsparadigm och operativsystem.
        \item Affärsrätt, datajuridik och interkulturell kommunikation.
    \end{itemize}
\end{cvevent}

\begin{cvevent}[2012][2015]
    \cvname{Naturvetarprogrammet, inriktning Natur}
    \cvdescription{Åva Gymnasium, Täby}
\end{cvevent}
%------------------------------------------------------

\cvsection[menu]{Övriga erfarenhet} %------------------------------------------------------

\begin{cvevent}[2017][2020]
    \cvname{Webmaster och Sportchef}
    \cvdescription{Linköping Universitets Hockeyförening}
\end{cvevent}

\begin{cvevent}[2009][2014]
    \cvname{Hel- och halvplansdomare för juniorer}
    \cvdescription{Hockeydomare, Täby}
\end{cvevent}

%------------------------------------------------------

\cvsidebar %-----------------------------------------------------------------------------


\cvsection[contact]{Kontakt}  %----------------------------------------------------------

\begin{cvitem}[Envelope][4]
    \textbf{Mejl}\\
    \href{mailto:gustavsvennas@gmail.com}{\texttt{gustavsvennas@gmail.com}}
\end{cvitem}

\cvseparator[3]
\begin{cvitem}[Phone][4]
    \textbf{Mobil}\\
    \href{tel:+46702968266}{\texttt{070 296 82 66}}
\end{cvitem}

\cvseparator[3]
\begin{cvitem}[Home][4]
    \textbf{Adress}\\
    Tullingebergsvägen 8C\\ 146 45 Tullinge
\end{cvitem}

\cvseparator[3]
\begin{cvitem}[Globe][4]
    \textbf{LinkedIn}\\
    \href{https://www.linkedin.com/in/gustav-svennas-18b575159}{\texttt{Gustav Svennas}}
\end{cvitem}
%-----------------------------------------------------------


\cvsection[program]{Programmering}  %-------------------------------------------------

\begin{cvitem}
    C++ (Goda kunskaper)
\end{cvitem}

\begin{cvitem}
    C (Goda kunskaper)
\end{cvitem}

\begin{cvitem}
    Java (Goda kunskaper)
\end{cvitem}

\begin{cvitem}
    Python (Grundläggande kunskaper)
\end{cvitem}

\cvseparator
\begin{cvitem}
    Matlab (Grundläggande kunskaper)
\end{cvitem}

\cvseparator
\begin{cvitem}
    Bash Scripting (Nybörjare)
\end{cvitem}
%-----------------------------------------------------------

\cvsection[skills]{Kompetenser}  %-----------------------------------------------------------

\cvseparator
\begin{cvitem}
    Git (Grundläggande)
\end{cvitem}



\cvseparator
\begin{cvitem}
    Linux, Kubuntu och Mint (God erfarenhet)
\end{cvitem}

\cvseparator
\begin{cvitem}
    Windows 7 och 10 (God erfarenhet)
\end{cvitem}

\cvseparator
\begin{cvitem}
    MacOS (Grundläggande)
\end{cvitem}

\cvseparator
\begin{cvitem}
    Jira (Lite erfarenhet)
\end{cvitem}


\cvsection[languages]{Språk}  %--------------------------------------------------------

\cvskill{Svenska}{Modersmål}{1.0}
\cvskill{Engelska}{Flytande}{0.95}
%--------------------------------------------------------


\cvsection[dance][1.8]{Intressen}  %--------------------------------------------------------

\cvseparator
\begin{cvitem}
    Golf
\end{cvitem}

\cvseparator
\begin{cvitem}
    Hockey 
\end{cvitem}

\cvseparator
\begin{cvitem}
    Vandring
\end{cvitem}

\cvseparator
\begin{cvitem}
    Social dans
\end{cvitem}

\cvseparator
\begin{cvitem}
    Bräd- och datorspel
\end{cvitem}


\end{cv}

\end{document}
