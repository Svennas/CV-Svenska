\documentclass{article}
\usepackage{bubblecv}

\begin{document}

%\pagecolor{lightgray}


\begin{cv}[ljus utsläppt sida 2]{Gustav Svennas}{Curriculum Vitae}


\cvsection[summary]{Profil}  %-----------------------------------------------------------
Jag är en sport- och teknikintresserad student som i september kommer ta en kandidat i datateknik (180hp). Som person har jag blivit beskriven som tålmodig, uppmärksam och hjälpsam. Så länge jag har kollegor så spelar det ingen roll om jag arbetar ensam eller i grupp. Jag har bl.a. spelat hockey i 20 år, där både individens och gruppens prestation är viktig. Tidigare anställningar kan bestyrka att jag är arbetsvillig, bra på att ta kritik och ofta försöker se det positiva i situationer, vilket gör mig enkel att arbeta med.


\cvsection[work]{Arbetslivserfarenhet} %------------------------------------------------------

\begin{cvevent}[2022-07][2022-08]
    \cvname{Kontorsservice}
    \cvdescription{Riksidrottsförbundet, Stockholm}
    \begin{itemize}
        \item Ansvarig för underhåll av kontoret.
    \end{itemize}
\end{cvevent}


\begin{cvevent}[2021-03][2021-10]
    \cvname{Uppdatering av inventariesystemen}
    \cvdescription{Klarna (IT-konsult via Academic Work), Stockholm}
    \begin{itemize}
        \item Ansvar för att uppdatera flera inventariesystem för anställdas datorer, mobiler och andra enheter.
        \item Samarbetade och hade kontakt med kollegor från bland annat USA, Tyskland och Australien.
    \end{itemize}
\end{cvevent}


\begin{cvevent}[2015-06][2015-07]
    \cvname{Produktionsarbetare}
    \cvdescription{Maquet Critical Care AB, Solna}
    \begin{itemize}
        \item Hjälpte till att förbereda delar för hopsättning av diverse medicinteknisk utrustning.
    \end{itemize}
\end{cvevent}
%------------------------------------------------------

\cvsection[education]{Utbildning}  %------------------------------------------------------

\begin{cvevent}[2023-04][2023-06]
    \cvname{Examensarbete, 16hp}
    \cvdescription{Wexnet, Växjö, via Linköpings Universitet}
    \begin{itemize}
        \item Studie där versionshanteringsverktygen GitHub, Azure Repos, Bitbucket och GitLab jämfördes.
    \end{itemize}
\end{cvevent}

\begin{cvevent}[2018-09][2023-06]
    \cvname{Kandidatprogram, Datateknik}
    \cvdescription{Linköpings Universitet (studiepaus 2021)}
    \textbf{Kurser:}
    \begin{itemize}
        \item Digitalteknik, datorkonstruktion, programutvecklingsmetodik, AI, cybersäkerhet m.m.
        \item Objektorienterad- och processprogrammering. 
        \item Algoritmer, programmeringsparadigm och operativsystem.
    \end{itemize}
\end{cvevent}
%----------------------------------------

\cvsection[menu]{Övriga erfarenhet} %------------------------------------------------------

\begin{cvevent}[2017-06][2020-06]
    \cvname{Webmaster och Sportchef}
    \cvdescription{Linköping Universitets Hockeyförening}
\end{cvevent}

%------------------------------------------------------

\cvsidebar %-----------------------------------------------------------------------------


\cvsection[contact]{Kontakt}  %----------------------------------------------------------

\begin{cvitem}[Envelope][4]
    \textbf{Mejl}\\
    \href{mailto:gustavsvennas@gmail.com}{\texttt{gustavsvennas@gmail.com}}
\end{cvitem}

\cvseparator[3]
\begin{cvitem}[Globe][4]
    \textbf{LinkedIn}\\
    \href{https://www.linkedin.com/in/gustav-svennas-18b575159}{\texttt{Gustav Svennas}}
\end{cvitem}

\cvseparator[3]
\begin{cvitem}[Phone][4]
    \textbf{Mobil}\\
    \href{tel:+46702968266}{\texttt{070 296 82 66}}
\end{cvitem}

\cvseparator[3]
\begin{cvitem}[Home][4]
    \textbf{Adress}\\
    Tullingebergsvägen 8C\\ 146 45 Tullinge
\end{cvitem}

\cvseparator[3]
\begin{cvitem}[Globe][4]
    \textbf{GitHub}\\
    \href{https://github.com/Svennas}{\texttt{Svennas}}
\end{cvitem}

%-----------------------------------------------------------


\cvsection[program]{Programmering}  %-------------------------------------------------

\begin{cvitem}
    C++ (Goda kunskaper)
\end{cvitem}

\begin{cvitem}
    C (Goda kunskaper)
\end{cvitem}

\begin{cvitem}
    Java (Goda kunskaper)
\end{cvitem}

\begin{cvitem}
    Python (Grundläggande kunskaper)
\end{cvitem}

\cvseparator
\begin{cvitem}
    Matlab (Grundläggande kunskaper)
\end{cvitem}

\cvseparator
\begin{cvitem}
    Bash Scripting (Nybörjare)
\end{cvitem}
%-----------------------------------------------------------

\cvsection[skills]{Kompetenser}  %-----------------------------------------------------------

\cvseparator
\begin{cvitem}
    Git (Grundläggande)
\end{cvitem}



\cvseparator
\begin{cvitem}
    Linux, Kubuntu och Mint (God erfarenhet)
\end{cvitem}

\cvseparator
\begin{cvitem}
    Windows 7 och 10 (God erfarenhet)
\end{cvitem}

\cvseparator
\begin{cvitem}
    MacOS (Grundläggande)
\end{cvitem}

\cvseparator
\begin{cvitem}
    Jira (Lite erfarenhet)
\end{cvitem}

\cvseparator
\begin{cvitem}
    B-körkort
\end{cvitem}


\cvsection[languages]{Språk}  %--------------------------------------------------------

\cvskill{Svenska}{Modersmål}{1.0}
\cvskill{Engelska}{Flytande}{0.95}
%--------------------------------------------------------


\cvsection[dance][1.8]{Intressen}  %--------------------------------------------------------

\cvseparator
\begin{cvitem}
    Golf
\end{cvitem}

\cvseparator
\begin{cvitem}
    Hockey 
\end{cvitem}

\cvseparator
\begin{cvitem}
    Vandring
\end{cvitem}

\cvseparator
\begin{cvitem}
    Social dans
\end{cvitem}

\cvseparator
\begin{cvitem}
    Bräd- och datorspel
\end{cvitem}


\end{cv}

\end{document}
